%%% A template to produce a nice-looking Curriculum Vitae.
%%% Original by Kieran Healy <kjhealy@gmail.com>, tweaked by Heather Miller <heather.c.miller@gmail.com>
%%% Most recent version is at http://github.com/heathermiller/cv
%%%
%%% ------------------------------------------------------------------------
%%% Requirements (should be included in a modern tex distribution):
%%% ------------------------------------------------------------------------
%%% xelatex
%%% fontspec.sty
%%% hyperrref.sty
%%% xunicode.sty
%%% color.sty
%%% url.sty
%%% fancyhdr.sty
%%%
%%% ------------------------------------------------------------------------
%%% Optional
%%% ------------------------------------------------------------------------
%%% git
%%% vc.sty
%%% revnum.sty
%%% Fonts
%%%
%%% ------------------------------------------------------------------------
%%% Note
%%%------------------------------------------------------------------------
%%% Because this is a hand-tweaked file, be on the look out for \medksip,
%%% \bigskip and \newpage commands here and there, which are used to balance
%%% the layout or avoid widows & orphans, etc. You should of course add or
%%% remove these as needed.
%%%------------------------------------------------------------------------

\documentclass[9pt]{article}

%%%------------------------------------------------------------------------
%%% Metadata
%%%------------------------------------------------------------------------

%% Change as needed. Or just add me as a coauthor. Only some of these are
%% used below in the hyperref declaration and address banner section.
\def\myauthor{Daniel Klug}
\def\mytitle{Vita}
\def\mycopyright{\myauthor}
\def\mykeywords{}
\def\mybibliostyle{plain}
\def\mybibliocommand{}
\def\mysubtitle{}
\def\myaffiliation{University of Basel}
\def\myaddress{Department of Philosophy \& Media Studies \\ Institute for Media Studies}
\def\myemail{Daniel.Klug@unibas.ch}
\def\mywebtext{https://populaerkultur.unibas.ch}
\def\myweb{https://populaerkultur.unibas.ch/home/popularculture-en/}
\def\myfax{+41 61 267 08 90}
\def\myphone{+41 78 696 40 58}
\def\myversion{}
\def\myrevision{}


\def\myaffiliation{University of Basel}
\def\myauthor{Daniel Klug}
\date{} % not used (revision control instead)
\def\mykeywords{Daniel, Klug, Daniel Klug, Vita, CV, Resume}

%%%------------------------------------------------------------------------
%%% Git version tracking
%%%------------------------------------------------------------------------

%% If you don't use git or the vc package (from CTAN), comment this out.
%% If you comment it out, be sure to remove the \rfoot comment below, too.
% \immediate\write18{sh ./vc}
% %%% This file has been generated by the vc bundle for TeX.
%%% Do not edit this file!
%%%
%%% Define Git specific macros.
\gdef\GITHash{9498a8f9e006690944554b09c3e15146911a9341}%
\gdef\GITAbrHash{9498a8f}%
\gdef\GITParentHashes{2e2b8a35d68d79c6a9211eafeafcf34900ace5ab}%
\gdef\GITAbrParentHashes{2e2b8a3}%
\gdef\GITAuthorName{Heather Miller}%
\gdef\GITAuthorEmail{heather.miller@epfl.ch}%
\gdef\GITAuthorDate{2014-02-13 17:20:12 +0100}%
\gdef\GITCommitterName{Heather Miller}%
\gdef\GITCommitterEmail{heather.miller@epfl.ch}%
\gdef\GITCommitterDate{2014-02-13 17:20:12 +0100}%
%%% Define generic version control macros.
\gdef\VCRevision{\GITAbrHash}%
\gdef\VCAuthor{\GITAuthorName}%
\gdef\VCDateRAW{2014-02-13}%
\gdef\VCDateISO{2014-02-13}%
\gdef\VCDateTEX{2014/02/13}%
\gdef\VCTime{17:20:12 +0100}%
\gdef\VCModifiedText{\textcolor{red}{with local modifications!}}%
%%% Assume clean working copy.
\gdef\VCModified{0}%
\gdef\VCRevisionMod{\VCRevision}%


%%%------------------------------------------------------------------------
%%% Required style files
%%%------------------------------------------------------------------------

\usepackage{url,fancyhdr}
\usepackage[ampersand]{easylist}
%%\usepackage{revnum} % for reverse-numbered publications (revnumerate environment) if needed.

%% needed for xelatex to work
\usepackage{fontspec}
\usepackage{xunicode}

%% color for the links
% \usepackage[usenames,dvipsnames]{color}
\usepackage[usenames,dvipsnames]{xcolor}
\definecolor{DarkBlue}{HTML}{265B8C}

%% hyperlinks
\usepackage[xetex,
	colorlinks=true,
	urlcolor=DarkBlue,
	plainpages=false,
  	pdfpagelabels,
  	bookmarksnumbered,
  	pdftitle={\mytitle},
  	pagebackref,
  	pdfauthor={\myauthor},
  	pdfkeywords={\mykeywords}
  	]{hyperref}

\usepackage{marvosym}

% \usepackage{showframe}
\usepackage[width=4.825in,top=1.7in]{geometry}

%%%------------------------------------------------------------------------
%%% Document
%%%------------------------------------------------------------------------
\begin{document}

%% Choose fonts for use with xelatex
%% Minion and Myriad are widely available, from Adobe.
%% Pragmata is available to buy at http://www.fsd.it/fonts/pragma.htm
%% and is worth every penny. Any good monospace font will work fine, though.
%% Consolas or inconsolata are good alternatives.
\setromanfont[Mapping={tex-text},Numbers={OldStyle},Ligatures={Common}]{Minion Pro}
\setsansfont[Mapping=tex-text,Colour=AA0000]{Myriad Pro}
\setmonofont[Mapping=tex-text,Scale=0.9]{Inconsolata}


%%%------------------------------------------------------------------------
%%% Local commands
%%%------------------------------------------------------------------------

%% Marginal header
%% Note: as the document goes on you may need to introduce a (gradually increasing)
%% \vspace element to keep the marginal header pleasingly aligned with the first
%% item in the body text. Like this: \marginhead{{\vskip 0.4em}Grants}, or
%% \marginhead{{\vskip 0.8em}Service}. Experiment as needed.
\newcommand{\marginhead}[1]{\marginpar{\textsf{{\normalsize\vspace{-1em}\flushright #1}}}}

\newcommand{\dates}[1]{\hfill \emph{#1}}

%% custom ampersand (font consistent with the one chosen above)
\newcommand{\amper}{{\fontspec[Scale=.95,Colour=AA0000]{Minion Pro Medium}\selectfont\&\,}}

%% No bullets on labels
\renewcommand{\labelitemi}{~}

%% Custom hanging indent for vita items
\def\ind{\hangindent=1 true cm\hangafter=1 \noindent}
%\def\ind{\hangindent=18pt\hangafter=1 \noindent}
\def\labelitemi{~}
\renewcommand{\labelitemii}{~}

%%%------------------------------------------------------------------------
%%% Page layout
%%%------------------------------------------------------------------------
\pagestyle{fancy}
\renewcommand{\headrulewidth}{0pt}
\fancyhead{}
\fancyfoot{}
\rhead{{\scriptsize\thepage}}

% \setlength{\headsep}{12pt}
\textheight=580pt
\raggedbottom
\thispagestyle{fancy}

%% git revision control footer
% \rfoot{\texttt{\scriptsize \VCRevision\ on \VCDateTEX}} % git revision info inserted via external script -- see docs for vc package for details. comment out this line if you're not using vc, and also remove the %%% This file has been generated by the vc bundle for TeX.
%%% Do not edit this file!
%%%
%%% Define Git specific macros.
\gdef\GITHash{9498a8f9e006690944554b09c3e15146911a9341}%
\gdef\GITAbrHash{9498a8f}%
\gdef\GITParentHashes{2e2b8a35d68d79c6a9211eafeafcf34900ace5ab}%
\gdef\GITAbrParentHashes{2e2b8a3}%
\gdef\GITAuthorName{Heather Miller}%
\gdef\GITAuthorEmail{heather.miller@epfl.ch}%
\gdef\GITAuthorDate{2014-02-13 17:20:12 +0100}%
\gdef\GITCommitterName{Heather Miller}%
\gdef\GITCommitterEmail{heather.miller@epfl.ch}%
\gdef\GITCommitterDate{2014-02-13 17:20:12 +0100}%
%%% Define generic version control macros.
\gdef\VCRevision{\GITAbrHash}%
\gdef\VCAuthor{\GITAuthorName}%
\gdef\VCDateRAW{2014-02-13}%
\gdef\VCDateISO{2014-02-13}%
\gdef\VCDateTEX{2014/02/13}%
\gdef\VCTime{17:20:12 +0100}%
\gdef\VCModifiedText{\textcolor{red}{with local modifications!}}%
%%% Assume clean working copy.
\gdef\VCModified{0}%
\gdef\VCRevisionMod{\VCRevision}%
 line above.


%%%------------------------------------------------------------------------
%%% Address and contact block
%%%------------------------------------------------------------------------
% \begin{absolutelynopagebreak}
\begin{minipage}[t]{2.95in}
 \flushright {\footnotesize \href{http://mewi.unibas.ch}{Department of Philosophy \& Media Studies, \\ \vspace{-0.03in} Institute for Media Studies} \\ University of Basel \\ Holbeinstrasse 12 \\ \vspace{-0.03in} 4051 Basel \\ \vspace{-0.05in} Switzerland}

\end{minipage}
\hfill
%\begin{minipage}[t]{0.0in}
% dummy (needed here)
%\end{minipage}
\hfill
\begin{minipage}[t]{1.7in}
  \flushright \footnotesize Phone: \myphone \\
  Fax: \myfax  \\
  {\scriptsize  \texttt{\href{mailto:\myemail}{\myemail}}} \\
  {\scriptsize  \vspace{-0.03in} \texttt{\href{\myweb}{\mywebtext}}}
\end{minipage}


\medskip

%% Name
\noindent{\huge {\textsc{daniel klug}}}
\reversemarginpar

\medskip

%% Citizenship
\medskip
\marginhead{Citizenship}

\noindent Germany
\newline\noindent {\em US permanent residence in process}

\bigskip


%% Research Interests
\marginhead{{\vskip 0.3em}Research \newline Interests}
\noindent Reality television, music television and music videos, audiovisual media products, computer-based media analyses, video sharing platforms, social network sites, popular music, qualitative media research, qualitative sociology

\bigskip

%% Education
\marginhead{Education}

\noindent{\bf \em University of Basel}, \emph{Basel, Switzerland} \vspace{0.01in} \dates{2008 -- 2012}
\newline Ph.D. in Media Studies, {\em summa cum laude}
\newline Advisor: Klaus Neumann-Braun
\bigskip

\noindent{\bf \em University of Vienna}, \emph{Vienna, Austria} \vspace{0.01in}  \dates{2002 -- 2008}
\newline\noindent Master in Sociology, minor in Theater, Film, \& Media Studies, Cultural Studies
\bigskip


%% Research Experience
\medskip
\marginhead{{\vskip 0.3em}Research \newline Experience}

\noindent {\bf Assistent}, {\bf \em University of Basel}, \emph{Basel, Switzerland} \vspace{0.01in} \dates{3/2012 -- }
\newline\noindent {\em Equivalent of Research Assistant Professor in US system.}
\newline\noindent with Klaus Neumann-Braun \vspace{0.3em} 
\newline\noindent {\bf {\em Varieties of scripted reality programs in television}} \dates{2014 -- 2016}
\newline\noindent and on the internet. Comparative analyses of production, 
\newline\noindent product and reception in (German-speaking) Switzerland. 
\newline\noindent Funded by the Swiss National Science Foundation.
\bigskip

\noindent {\bf Doctoral Student}, {\bf \em University of Basel}, \emph{Basel, Switzerland} \vspace{0.01in} \dates{7/2008 -- 3/2012}
\newline\noindent with Klaus Neumann-Braun \vspace{0.3em} 
\newline\noindent {\bf {\em Image-Text-Sound-Analyses of music videos.}} \dates{2008 -- 2011}
\newline\noindent Analysis of audiovisual relationships in music videos. 
\newline\noindent Developed an analytical tool (trAVis) for music-centered 
\newline\noindent transcription and analysis of audiovisual clips.
\newline\noindent Funded by the Swiss National Science Foundation.
\bigskip


% Analyses of the production methods of fictional reality shows (scripted reality) using qualitative methods (expert interviews, participatory observations).
% Analyses of the specific audio visual relations between moving images and sound in music videos including development of a computer-based tool for the transcription of short video clips


%% Publications
\marginhead{Publications: Books}
% \medskip

%% Use revnumerate environment if numbered publications are needed.
%% (Include it above in the preamble).
%% \renewcommand{\labelenumi}{\textsc{a}\theenumi.}
%% \begin{revnumerate}

% \noindent\textbf{Books/Monographs}

\noindent\textbf{Lip synching in Music Videos. On the Construction of Audiovision}\dates{2013}\vspace{-0.03in}
\newline\noindent\textbf{Through Music-Related Representational Acts}\dates{}
\newline\noindent{\em Lip Synching in Musikclips. Zur Konstruktion von Audio-Vision}\dates{}\vspace{-0.03in}
\newline\noindent{\em durch musikbezogene Darstellungshandlungen}
\newline\noindent Daniel Klug
\newline\noindent\emph{(Reihe Short Cuts | Cross Media, Band 6). Baden-Baden: Nomos.}
% \newline\noindent\emph{and Concurrency Centric Systems}
\bigskip


\noindent\textbf{Computer-Based Analysis of Audiovisual Media Artifacts}\dates{2013}
\newline\noindent{\em Computergest\"{u}tzte Analyse von audiovisuellen Medienprodukten}\dates{}
\newline\noindent Christofer Jost, Daniel Klug, Axel Schmidt, Armin Reautschnig, Klaus Neumann-Braun
\newline\noindent\emph{(Reihe Qualitative Sozialforschung, Band 22). Wiesbaden: Springer VS.}
% \newline\noindent\emph{and Concurrency Centric Systems}
\bigskip


\marginhead{Publications: Edited Books}

\noindent{\bf PPopular Music, Medial Music? Transdisciplinary Contributions on}\dates{2011}\vspace{-0.03in}
\newline\noindent{\bf the Media of Popular Music}
\newline\noindent{\em Popul\"{a}re Musik, mediale Musik? Transdisziplin\"{a}re Beitr\"{a}ge zu}\dates{2011}\vspace{-0.03in}
\newline\noindent{\em den Medien der popul\"{a}ren Musik}
% \smallskip
\newline\noindent Christofer Jost, Daniel Klug, Axel Schmidt, Klaus Neumann-Braun
\newline\noindent\emph{(Reihe Short Cuts | Cross Media, Band 3). Baden-Baden: Nomos.}
\bigskip

\noindent{\bf The Meaning of Popular Music in Audiovisual Artifacts}\dates{2009}
\newline\noindent {\em Die Bedeutung populärer Musik in audiovisuellen Formaten}
% \smallskip
\newline\noindent Christofer Jost, Klaus Neumann-Braun, Daniel Klug, Axel Schmidt
\newline\noindent\emph{(Reihe Short Cuts | Cross Media, Band 1). Baden-Baden: Nomos.}
\bigskip

\marginhead{Publications: Journal Articles}

\noindent{\bf Computer-based Analysis of Audiovisual Media Artifacts in}\dates{2015}\vspace{-0.03in}
\newline\noindent{\bf School Music Lessons, In: Teaching Music in the 21st Century (in print)}
\newline\noindent {\em Computergest\"{u}tzte Analyse von audiovisuellen Medienangeboten}\vspace{-0.03in}
\newline\noindent{\em im schulischen Musikunterricht}
% \smallskip
\newline\noindent Daniel Klug
\newline\noindent\emph{Musikunterricht(en) im 21. Jahrhundert. Augsburg (i. Ersch.).}
\bigskip

\noindent{\bf Scripted Reality-Shows in German-speaking Television Programs.}\dates{2014}\vspace{-0.03in}
\newline\noindent{\bf Tri-National Program Analysis and the Concept of a Combined Analysis}\vspace{-0.03in}
\newline\noindent{\bf of Product and Production}
\newline\noindent {\em Scripted Reality-Formate im deutschsprachigen Fernsehprogramm.}\vspace{-0.03in}
\newline\noindent {\em Trinationale Programmanalyse und Konzeption einer kombinierten}\vspace{-0.03in}
\newline\noindent {\em Produkt- und Produktionsanalyse}
% \smallskip
\newline\noindent Daniel Klug, Axel Schmidt
\newline\noindent\emph{In: Studies in Communication Sciences, 14(2014), S. 108--120.}
\bigskip

\noindent{\bf The Body (and Its Representations) in Factual Entertainment.}\dates{2014}\vspace{-0.03in}
\newline\noindent{\bf  Producing Reality in and Beyond Television}
\newline\noindent {\em K\"{o}rper(-Darstellungen) im Reality-TV. Herstellung von Wirklichkeit}\vspace{-0.03in} 
\newline\noindent {\em im und über das Fernsehen hinaus}
% \smallskip
\newline\noindent Daniel Klug, Axel Schmidt
\newline\noindent\emph{In sozialer sinn, 1/2014. S. 77--107.}
\bigskip


\noindent{\bf A Song For the Lovers}\dates{2011}
% \smallskip
\newline\noindent Daniel Klug
\newline\noindent\emph{In: Fischer, Michael/Grosch, Nils/Hörner, Fernand: Songlexikon.}
\newline\noindent URL: \href{http://www.songlexikon.de/songs/asongforthelovers}{http://www.songlexikon.de/songs/asongforthelovers}
\bigskip


\noindent{\bf (In-)Coherences. On the Performative Practice of Lip Synching in}\dates{2011}\vspace{-0.03in}
\newline\noindent{\bf the Audio-Vision of Music Videos}
\newline\noindent{\em (Un-)Stimmigkeiten. Zur Darstellungspraxis des lip synching in}\vspace{-0.03in}
\newline\noindent{\em der Audio-Vision des Musikclips} 
% \smallskip
\newline\noindent Christofer Jost, Daniel Klug, Axel Schmidt, Klaus Neumann-Braun
\newline\noindent\emph{Popul\"{a}re Musik, mediale Musik? Transdisziplin\"{a}re Beitr\"{a}ge zu den Medien}\vspace{-0.03in}
\newline\noindent\emph{der popul\"{a}ren Musik}
\newline\noindent\emph{(Reihe Short Cuts | Cross Media, Band 3). Baden-Baden: Nomos. S. 7--29.}
\bigskip

\pagebreak

\noindent{\bf Two Become One? The Original in the Audio-Vision of the Music Video}\dates{2011}
\newline\noindent{\em Aus zwei mach eins? Das Original(e) in der Audio-Vision des Musikclips} 
% \smallskip
\newline\noindent Daniel Klug
\newline\noindent\emph{Lied und populäre Kultur /Song and Popular Culture. Jahrbuch des}\vspace{-0.03in}
\newline\noindent\emph{Deutschen Volksliedarchivs. 56. Jahrgang: Original und Kopie/ Original and Copy.}
\newline\noindent\emph{Münster: Waxmann. S. 43--61.}
\bigskip


\noindent{\bf The Monstrous Body in Music Videos}\dates{2011}
\newline\noindent{\em Der monstr\"{o}se K\"{o}rper im Musikclip} 
% \smallskip
\newline\noindent Daniel Klug
\newline\noindent\emph{Dawn of an Evil Millennium. Horror/ Kultur im neuen Jahrtausend.}
\newline\noindent\emph{Darmstadt: Büchner. S. 312–318.}
\bigskip

\noindent{\bf All eyes on... music? Music and Audio Vision in Transition}\dates{2011}
\newline\noindent{\em All eyes on... music? Musik und Audiovision im Wandel} 
% \smallskip
\newline\noindent Daniel Klug, Klaus Neumann-Braun
\newline\noindent\emph{Imageb(u)ilder. Vergangenheit, Gegenwart und Zukunft des Videoclips}
\newline\noindent\emph{(Ausstellungsband). Münster: Telos, S. 52--71.}
\bigskip


\noindent{\bf ...dont be afraid, don't have no fear -- Horror Aesthetics in the Pop}\dates{2010}\vspace{-0.03in}
\newline\noindent{\bf Music Video ``Everybody (Backstreet's back)''}
\newline\noindent{\em '...don’t be afraid, don’t have no fear' – Horror\"{a}sthetik im Popmusikclip}\vspace{-0.03in}
\newline\noindent{\em zu 'Everybody (Backstreet’s back)'} 
% \smallskip
\newline\noindent Daniel Klug
\newline\noindent\emph{pop:aesthetiken. Beitr\"{a}ge zum Sch\"{o}nen in der popul\"{a}ren Musik}
\newline\noindent\emph{(Werkstatt Popul\"{a}re Musik, Band 2). Innsbruck: Studienverlag. S. 139–161.}
\bigskip

\noindent{\bf Integrated Analysis of Image, Text, and Sound. The Example of the}\dates{2009}\vspace{-0.03in}
\newline\noindent{\bf Music Video ``Californication''}
\newline\noindent{\em Integrierte Bild-Text-Ton-Analyse. Am Beispiel des Musikclips 'Californication'}
% \smallskip
\newline\noindent Christofer Jost, Daniel Klug
\newline\noindent\emph{Die Bedeutung popul\"{a}rer Musik in audiovisuellen Formaten }
\newline\noindent\emph{(Reihe Short Cuts | Cross Media, Band 1). Baden-Baden: Nomos, S. 197--242.}
\bigskip


\noindent{\bf The Song in the Context of Audio Vision. Introduction to a}\dates{2009}\vspace{-0.03in}
\newline\noindent{\bf Dispersed Research Area}
\newline\noindent{\em Der Song im Zeichen der Audiovision. Zur Einf\"{u}hrung in ein disparates Forschungsfeld}
% \smallskip
\newline\noindent Christofer Jost, Klaus Neumann-Braun, Daniel Klug, Axel Schmidt
\newline\noindent\emph{Die Bedeutung popul\"{a}rer Musik in audiovisuellen Formaten }
\newline\noindent\emph{(Reihe Short Cuts | Cross Media, Band 1). Baden-Baden: Nomos, S. 7--19.}
\bigskip


%% Research Reports
\marginhead{Research Reports}

\noindent{\bf The Pedagogical Concept of Peer Education in the Context of}\dates{2012}\vspace{-0.03in}
\newline\noindent{\bf Media Literacy Advancement and Youth Media Protection}
\newline\noindent{\em Das p\"{a}dagogische Konzept der Peer Education im Rahmen von}\vspace{-0.03in}
\newline\noindent{\em Medienkompetenzf\"{o}rderung und Jugendmedienschutz}
% \smallskip
\newline\noindent Klaus Neumann-Braun, Vanessa Kleinschnittger, Michael Baumg\"{a}rtner,\vspace{-0.03in}
\newline\noindent Daniel Klug, Alessandro Preite, Luca Preite
\newline\noindent (Berichtnummer 15/12).
\bigskip


\pagebreak

\noindent{\bf Risk Factors in Young People’s Use of Digital Media and Possible }\dates{2012}\vspace{-0.03in}
\newline\noindent{\bf Strategies in the Context of Prevention and Intervention}
\newline\noindent{\em Risikofaktoren bei der Nutzung digitaler Medien durch Jugendliche und m\"{o}gliche}\vspace{-0.03in}
\newline\noindent{\em Handlungsstrategien im Rahmen von Pr\"{a}vention und Intervention}
% \smallskip
\newline\noindent Klaus Neumann-Braun, Vanessa Kleinschnittger, Michael Baumg\"{a}rtner,\vspace{-0.03in}
\newline\noindent Daniel Klug, Alessandro Preite, Luca Preite
\newline\noindent (Berichtnummer 12/12).
\bigskip


%% Presentations
% \medskip
\marginhead{Presentations}


\vspace{-0.02in}
\noindent{\bf trAVis – A Tool for Mulitmodal Data Analysis} \dates{2015}
\linebreak\noindent 2nd Bremen Conference on Multimodality (BreMM15). \dates{}
\linebreak\noindent Bremen, Germany. September 21-22, 2015.
\bigskip


\noindent{\bf Productions of Scripted Reality Between Fact and Fiction} \dates{2015}
\linebreak\noindent {\em Scripted Reality-Produktionen zwischen Fakt und Fiktion}\dates{}
\linebreak\noindent The (Un-)Truth of Images. Modes of Reality in TV and Cinema Conference. \dates{}
\linebreak\noindent Kiel, Germany. March 26-28, 2015.
\bigskip

\noindent{\bf The Construction of Images Including a Capability for Truth -- Production}\dates{2015}\vspace{-0.03in}
\linebreak\noindent{\bf Techniques in Factual Entertainment} \dates{}
\linebreak\noindent {\em Die Herstellung wahrheitsfähiger Bilder – Produktionstechniken}\dates{}\vspace{-0.03in}
\linebreak\noindent {\em im factual entertainment}\dates{}
\linebreak\noindent The (Un-)Truth of Images. Modes of Reality in TV and Cinema Conference. \dates{}
\linebreak\noindent Kiel, Germany. March 26-28, 2015.
\bigskip


\noindent{\bf Displaying the Self through Moving Images. Functions and Consequences of}\dates{2014}\vspace{-0.03in}
\linebreak\noindent{\bf YouTube-Videos on Peer-Perception and Self-Identification}\dates{}
\linebreak\noindent 64th Annual Conference of the International Communications Association (ICA).\dates{}
\linebreak\noindent Seattle, WA, USA. May 21-25, 2014.
\bigskip

\noindent{\bf Entertainment Television Between Fact and Fiction: Comparative Analysis of}\dates{2014}\vspace{-0.03in}
\linebreak\noindent{\bf the Product, Production, and Perception of Scripted Reality Shows}\dates{}
\linebreak\noindent{\em Zwischen fiktionaler und faktualer Fernsehunterhaltung: Vergleichende}\dates{}\vspace{-0.03in}
\linebreak\noindent{\em Produktions-, Produkt- und Rezeptionsanalysen von Scripted Reality-Formaten}\dates{}
\linebreak\noindent Conference of the Swiss Association for Communication and Media Research.\dates{}
\linebreak\noindent Zurich, Switzerland. April 11-12, 2014.
\bigskip


\noindent{\bf What's that sound? Creating realistic audio visual experiences in music videos}\dates{2013}
\linebreak\noindent 1st International Conference of the European Sound Studies Association (ESSA). \dates{}
\linebreak\noindent Berlin, Germany. October 11-12, 2014.
\bigskip


\noindent{\bf Reality TV and the Construction of Realness in Television and Beyond}\dates{2013}
\linebreak\noindent{\em Reality TV – Herstellung von Wirklichkeit im und über das Fernsehen hinaus}\dates{}
\linebreak\noindent Annual Conference of the History of Television/Television Studies Division of\dates{}\vspace{-0.03in} 
\linebreak\noindent the Society for Media Studies.\dates{}
\linebreak\noindent Regensburg, Germany. May 3--5, 2013.
\bigskip


\noindent{\bf trAVis – A Music-centered Transcription Tool for Audiovisual Media Artifacts}\dates{2013}
\linebreak\noindent{\em trAVis - Musikzentriertes Transkriptionsprogramm f\"{u}r audiovisuelle Medienprodukte}\dates{}
\linebreak\noindent 17th Work Conference for Conversational Research.\dates{}
\linebreak\noindent Institute for German Language, Mannheim, Germany. March 22, 2013.
\bigskip


\pagebreak

\noindent{\bf Computer-based Analysis of Music Videos}\dates{2013}
\linebreak\noindent{\em Computergest\"{u}tzte Analyse von Musikclips}\dates{}
\linebreak\noindent University of Applied Science North-Western Switzerland.\dates{}
\linebreak\noindent Aarau, Switzerland. February 12, 2013.
\bigskip


\noindent{\bf Analyzing Television Products Using the Transcription Tool trAVis:}\dates{2013}\vspace{-0.03in}
\linebreak\noindent{\bf Workflow, Variations and Comparisons}\dates{}
\linebreak\noindent{\em Analyse von Fernsehprodukten mit trAVis – Workflow, Variationen und Vergleiche}\dates{}
\linebreak\noindent Music – Computer – Analysis Conference.\dates{}
\linebreak\noindent Basel, Switzerland. February 8-9, 2013.
\bigskip


\noindent{\bf trAVis - Transcribing Music-Based Audiovisual Media Through}\vspace{-0.03in}
\dates{2012}
\linebreak\noindent{\bf Computer-based Analysis}\dates{}
\linebreak\noindent 7th Conference of the European Research Network ``Sociology of the Arts''.\dates{}
\linebreak\noindent Vienna, Austria. September 5-8, 2012.
\bigskip


\noindent{\bf Lip Synching As Performance Strategy in Music Videos}
\dates{2011}
\linebreak\noindent{\em Lip Synching als Darstellungsstrategie im Musikclip. Zur Konstruktion von Audiovision}
\linebreak\noindent Basel PhD-Colloquium ``Methods and Research''.\dates{}
\linebreak\noindent Basel, Switzerland. April 12, 2011.
\bigskip


\noindent{\bf Computer-based Integrated Analysis of Images, Text, and Sound in}\vspace{-0.03in}
\dates{2011}
\linebreak\noindent{\bf Audio Visual Media Artifacts}\dates{}
\linebreak\noindent{\em Computergest\"{u}tzte integrierte Bild-Text-Ton-Analyse audiovisueller Materialien}\dates{}
\linebreak\noindent Conference of the Swiss Association for Communication and Media Research.\dates{}
\linebreak\noindent Basel, Switzerland. April 8-9, 2011.
\bigskip

\noindent{\bf Visualization of Popular Music in Music Videos}
\dates{2010}
\linebreak\noindent{\em Visualisierung popul\"{a}rer Musik im Musikclip}\dates{}
\linebreak\noindent Annual Conference of the Workgroup for the Study of Popular Music.\dates{}
\linebreak\noindent Mannheim, Germany. November 19-21, 2010.
\bigskip


\noindent{\bf Intermedia Structures in Music Videos. Towards an Integrated Analysis}\dates{2009}\vspace{-0.03in}
\linebreak\noindent{\bf of Images, Text, and Sound, Illustrated With the Music Video ``Californication''}\dates{}
\linebreak\noindent{\em Intermediale Strukturen im Musikclip. Auf dem Weg zu einer integrierten}\dates{}\vspace{-0.03in}
\linebreak\noindent{\em Bild-Text-Ton-Analyse, veranschaulicht am Musikclips 'Californication'}\dates{}
\linebreak\noindent  Workshop on the significance of popular music in audiovisual formats.\dates{}
\linebreak\noindent Basel, Switzerland. February 5-6, 2009.
\bigskip


\noindent{\bf Forms and Functions in the Presentation of Horror in Music Videos}\dates{2007}
\linebreak\noindent{\em Formen und Funktionen der Inszenierung von Horror in Musikvideoclips}\dates{} 
\linebreak\noindent Workshop and Workshow Visual Sociology.\dates{}
\linebreak\noindent Vienna, Austria. November 23-24, 2007.
\bigskip


\noindent{\bf The Brief Horror in Music Videos}\dates{2007}
\linebreak\noindent{\em Das kurze Grauen. Horror in Musikvideoclips}\dates{} 
\linebreak\noindent Conference of ``Project Intermediality''.\dates{}
\linebreak\noindent Vienna, Austria. March 30-31, 2007.
\bigskip

\pagebreak


%% Teaching Experience
\medskip
\marginhead{Teaching \newline Experience}

\noindent {\bf Supervision of Master \& Bachelor Theses}\dates{2009 --} 
\newline\noindent Bachelor \& Master-level, University of Basel. Basel, Switzerland.
\bigskip

% \begin{minipage}{\linewidth}
\noindent {\bf Computer-based Analysis of Audio Visual Media Artifacts Using the}\dates{2013 --}\vspace{-0.03in} 
\newline\noindent{\bf Web Application trAVis}\dates{}
\newline\noindent {\em Computergest\"{u}tzte Analyse audiovisueller Medienprodukte mit der}\vspace{-0.03in} 
\newline\noindent {\em Web- Applikation trAVis}
\newline\noindent Master-level course, University of Basel. Basel, Switzerland.
\bigskip

\noindent {\bf Ways of Staging Media Realities}\dates{Fall 2013} 
\newline\noindent {\em Arten und Weisen der Inszenierung medialer Wirklichkeiten} 
\newline\noindent Master-level course, University of Mannheim. Mannheim, Germany.
\bigskip

\noindent {\bf Qualitative Methods of Television Studies}\dates{2012 --} 
\newline\noindent {\em Qualitative Methoden der Fernsehforschung} 
\newline\noindent Bachelor-level course, University of Basel. Basel, Switzerland.
\bigskip

\noindent {\bf Analysis of Short Audio Visual Clips On YouTube}\dates{Spring 2012} 
\newline\noindent {\em Analyse audiovisueller Kurzformate am Beispiel von YouTube} 
\newline\noindent Bachelor-level course, University of Basel. Basel, Switzerland.
\bigskip

\noindent {\bf Analysis of (Audio) Visual Communication: Reality-TV II}\dates{Fall 2011} 
\newline\noindent {\em Analyse (audio-)visueller Kommunikation: Reality-TV} 
\newline\noindent Bachelor-level course, University of Basel. Basel, Switzerland.
\bigskip


\noindent {\bf Analysis of (Audio) Visual Communication: Reality-TV}\dates{Spring 2011} 
\newline\noindent {\em Analyse (audio-)visueller Kommunikation: Reality-TV} 
\newline\noindent Bachelor-level course, University of Basel. Basel, Switzerland.
\bigskip


\noindent {\bf Research Course: Reality-TV}\dates{Spring 2010} 
\newline\noindent {\em Forschungswerkstatt Reality-TV} 
\newline\noindent Bachelor-level course, University of Basel. Basel, Switzerland.
\bigskip


\noindent {\bf Product Analyses and Videoclips}\dates{Fall 2009} 
\newline\noindent {\em Produktanalysen und Videoclips} 
\newline\noindent Bachelor-level course, University of Basel. Basel, Switzerland.
\bigskip


%% Conference Organization
\medskip
\marginhead{Conference \newline Organization}

\noindent {\bf PhD-Conference ``Working With Everyday Life--How Reality TV Creates Facts!'', }\dates{11/2012} 
\newline\noindent Institute for Media Studies, University of Basel. Basel, Switzerland.
\bigskip

\noindent {\bf Popular Music, Medial Music?}\dates{6/2010} 
\newline\noindent Institute for Media Studies, University of Basel. Basel, Switzerland.
\bigskip

\noindent {\bf The Meaning of Popular Music in Audio Visual Artifacts}\dates{2/2009} 
\newline\noindent Institute for Media Studies, University of Basel. Basel, Switzerland.
\bigskip

\pagebreak


%% Professional Memberships
\medskip
\marginhead{Professional \newline Memberships}
\begin{easylist}[itemize]
& European Sociological Association (ESA)

& Gesellschaft f\"{u}r Popularmusikforschung e.V. (GfPM)
\newline {\em Society for the Studying of Popular Music}

& International Communication Association (ICA)

& Schweizer Gesellschaft f\"{u}r Kommunikations- und Medienwissenschaft (SGKM)
\newline {\em Swiss Association of Communication and Media Research}
\end{easylist}
\bigskip

%% References
\bigskip
\marginhead{References}

\noindent {\bf Klaus Neumann-Braun}
\newline\noindent {Department of Philosophy and Media Studies}
\newline\noindent {\em University of Basel, Switzerland}
\newline\noindent \Telefon~+41 61 267 08 71
\newline\noindent \Letter~\href{mailto:K.Neumann-Braun@unibas.ch}{K.Neumann-Braun@unibas.ch}
\medskip

\noindent {\bf Axel Schmidt}
\newline\noindent {\em Institute for the German Language (IDS), Germany}
\newline\noindent \Telefon~+49 621 1581 317
\newline\noindent \Letter~\href{mailto:Axel.Schmidt@ids-mannheim.de}{Axel.Schmidt@ids-mannheim.de}
\medskip

\noindent {\bf Christofer Jost}
\newline\noindent {Centre for Popular Culture and Music (ZPKM)}
\newline\noindent {\em University of Freiburg, Germany}
\newline\noindent \Telefon~+49 761 70 503 18
\newline\noindent \Letter~\href{mailto:christofer.jost@zpkm.uni-freiburg.de}{christofer.jost@zpkm.uni-freiburg.de}
\medskip

\end{document}
