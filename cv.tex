%%% A template to produce a nice-looking Curriculum Vitae.
%%% Original by Kieran Healy <kjhealy@gmail.com>, tweaked by Heather Miller <heather.c.miller@gmail.com>
%%% Most recent version is at http://github.com/heathermiller/cv
%%%
%%% ------------------------------------------------------------------------
%%% Requirements (should be included in a modern tex distribution):
%%% ------------------------------------------------------------------------
%%% xelatex
%%% fontspec.sty
%%% hyperrref.sty
%%% xunicode.sty
%%% color.sty
%%% url.sty
%%% fancyhdr.sty
%%%
%%% ------------------------------------------------------------------------
%%% Optional
%%% ------------------------------------------------------------------------
%%% git
%%% vc.sty
%%% revnum.sty
%%% Fonts
%%%
%%% ------------------------------------------------------------------------
%%% Note
%%%------------------------------------------------------------------------
%%% Because this is a hand-tweaked file, be on the look out for \medksip,
%%% \bigskip and \newpage commands here and there, which are used to balance
%%% the layout or avoid widows & orphans, etc. You should of course add or
%%% remove these as needed.
%%%------------------------------------------------------------------------

\documentclass[9pt]{article}

%%%------------------------------------------------------------------------
%%% Metadata
%%%------------------------------------------------------------------------

%% Change as needed. Or just add me as a coauthor. Only some of these are
%% used below in the hyperref declaration and address banner section.
\def\myauthor{Heather Miller}
\def\mytitle{Vita}
\def\mycopyright{\myauthor}
\def\mykeywords{}
\def\mybibliostyle{plain}
\def\mybibliocommand{}
\def\mysubtitle{}
\def\myaffiliation{Carnegie Mellon University}
\def\myaddress{School of Computer Science}
\def\myemail{heather.miller@cs.cmu.edu}
\def\myweb{http://heather.miller.am}
%\def\myfax{+1 (617) 373-5121}
\def\myphone{+1 (646) 301-1825}
\def\myversion{}
\def\myrevision{}


\def\myaffiliation{CMU}
\def\myauthor{Heather Miller}
\date{} % not used (revision control instead)
\def\mykeywords{Heather, Miller, Heather Miller, Vita, CV, Resume, Scala, Programming Languages}

%%%------------------------------------------------------------------------
%%% Git version tracking
%%%------------------------------------------------------------------------

%% If you don't use git or the vc package (from CTAN), comment this out.
%% If you comment it out, be sure to remove the \rfoot comment below, too.
% \immediate\write18{sh ./vc}
% \input{vc}

%%%------------------------------------------------------------------------
%%% Required style files
%%%------------------------------------------------------------------------

\usepackage{url,fancyhdr}
\usepackage[ampersand]{easylist}
%%\usepackage{revnum} % for reverse-numbered publications (revnumerate environment) if needed.

%% needed for xelatex to work
\usepackage{fontspec}
\usepackage{xunicode}

%% color for the links
% \usepackage[usenames,dvipsnames]{color}
\usepackage[usenames,dvipsnames]{xcolor}
\definecolor{DarkBlue}{HTML}{265B8C}

%% hyperlinks
\usepackage[xetex,
	colorlinks=true,
	urlcolor=DarkBlue,
	plainpages=false,
  	pdfpagelabels,
  	bookmarksnumbered,
  	pdftitle={\mytitle},
  	pagebackref,
  	pdfauthor={\myauthor},
  	pdfkeywords={\mykeywords}
  	]{hyperref}

\usepackage{marvosym}

% \usepackage{showframe}
\usepackage[width=4.825in,top=1.7in]{geometry}

%%%------------------------------------------------------------------------
%%% Document
%%%------------------------------------------------------------------------
\begin{document}

%% Choose fonts for use with xelatex
%% Minion and Myriad are widely available, from Adobe.
%% Pragmata is available to buy at http://www.fsd.it/fonts/pragma.htm
%% and is worth every penny. Any good monospace font will work fine, though.
%% Consolas or inconsolata are good alternatives.
\setromanfont[Mapping={tex-text},Numbers={OldStyle},Ligatures={Common}]{Minion Pro}
\setsansfont[Mapping=tex-text,Colour=AA0000]{Myriad Pro}
\setmonofont[Mapping=tex-text,Scale=0.9]{Inconsolata}


%%%------------------------------------------------------------------------
%%% Local commands
%%%------------------------------------------------------------------------

%% Marginal header
%% Note: as the document goes on you may need to introduce a (gradually increasing)
%% \vspace element to keep the marginal header pleasingly aligned with the first
%% item in the body text. Like this: \marginhead{{\vskip 0.4em}Grants}, or
%% \marginhead{{\vskip 0.8em}Service}. Experiment as needed.
\newcommand{\marginhead}[1]{\marginpar{\textsf{{\normalsize\vspace{-1em}\flushright #1}}}}

\newcommand{\dates}[1]{\hfill \emph{#1}}

%% custom ampersand (font consistent with the one chosen above)
\newcommand{\amper}{{\fontspec[Scale=.95,Colour=AA0000]{Minion Pro Medium}\selectfont\&\,}}

%% No bullets on labels
\renewcommand{\labelitemi}{~}

%% Custom hanging indent for vita items
\def\ind{\hangindent=1 true cm\hangafter=1 \noindent}
%\def\ind{\hangindent=18pt\hangafter=1 \noindent}
\def\labelitemi{~}
\renewcommand{\labelitemii}{~}

%%%------------------------------------------------------------------------
%%% Page layout
%%%------------------------------------------------------------------------
\pagestyle{fancy}
\renewcommand{\headrulewidth}{0pt}
\fancyhead{}
\fancyfoot{}
\rhead{{\scriptsize\thepage}}

% \setlength{\headsep}{12pt}
\textheight=580pt
\raggedbottom
\thispagestyle{fancy}

%% git revision control footer
% \rfoot{\texttt{\scriptsize \VCRevision\ on \VCDateTEX}} % git revision info inserted via external script -- see docs for vc package for details. comment out this line if you're not using vc, and also remove the \input{vc} line above.


%%%------------------------------------------------------------------------
%%% Address and contact block
%%%------------------------------------------------------------------------
% \begin{absolutelynopagebreak}
\begin{minipage}[t]{2.95in}
 \flushright {\footnotesize \href{https://www.cs.cmu.edu/}{School of Computer Science} \\ \href{http://s3d.cmu.edu}{Software and Societal Systems Department} \\ Carnegie Mellon University \\ \vspace{-0.03in} 5000 Forbes Ave \\ \vspace{-0.03in} Pittsburgh, PA 15213 \\ \vspace{-0.05in} USA}

\end{minipage}
\hfill
%\begin{minipage}[t]{0.0in}
% dummy (needed here)
%\end{minipage}
\hfill
\begin{minipage}[t]{1.7in}
  \flushright \footnotesize Phone: \myphone \\
  %Fax: \myfax  \\
  {\scriptsize  \texttt{\href{mailto:\myemail}{\myemail}}} \\
  {\scriptsize  \vspace{-0.03in} \texttt{\href{\myweb}{\myweb}}}
\end{minipage}


\medskip

%% Name
\noindent{\huge {\textsc{heather miller}}}
\reversemarginpar

\medskip
\bigskip

 %% Citizenship
% \medskip
%  \marginhead{Citizenship}

%  \noindent USA

%  \medskip

%% Research Interests
\textheight=580pt
\marginhead{{\vskip 0.3em}Research \newline Interests}
% \medskip

\noindent Concurrent, distributed, eventually-consistent (edge computing), data-centric, and data-intensive (big data) programming, from the perspective of programming languages. More recently, my work has come to include {\em programming LLM systems}, or, focusing on how best to program \textit{\textbf{\href{https://bair.berkeley.edu/blog/2024/02/18/compound-ai-systems/}{Compound AI Systems}}}. I work on both theoretical ideas {\em \&} implementations. {\bf My goal is to reduce the burden of building distributed, and increasingly, AI-enabled systems.}
\bigskip


% Programming language support for concurrent and distributed programming; \\type systems; non-standard uses of types for data-centric programming and big data; language and library design


%% Education
\marginhead{Education}

\noindent{\bf \em EPFL}, \emph{Lausanne, Switzerland} \vspace{0.01in} \dates{2009 -- 2015}
\newline Ph.D. in Computer Science
\newline Advisor: Martin Odersky \dates{2011 -- 2015}

\bigskip

\noindent{\bf \em University of Miami}, \emph{Coral Gables, FL} \vspace{0.01in}  \dates{2006 -- 2009}
\newline\noindent BSEE in Electrical Engineering, Audio Engineering, {\em with honors, May 2009}

\medskip

\noindent{\bf \em Cooper Union for the Advancement of Science and Art}, \emph{New York, NY} \vspace{0.01in}  \dates{2004 -- 2006}

\bigskip

%% Employment
\medskip
\marginhead{Employment}

\noindent {\bf Two Sigma Investments}, \emph{New York City, NY, USA} \vspace{0.01in} \dates{10/2022 -- }
\newline\noindent {\bf \em Vice President, Research Scientist}
\newline Two Sigma Labs team, research interests: distributed programming,\vspace{-0.03in} 
\newline distributed systems, and programming LLMs.
\bigskip

\noindent {\bf Carnegie Mellon University}, \emph{Pittsburgh, PA, USA} \vspace{0.01in} \dates{8/2018 -- }
\newline\noindent {\bf \em Assistant Professor}
\newline School of Computer Science, Software and Societal Systems Department\vspace{-0.03in}
\newline Co-founder (with Ben L. Titzer) of the \textbf{\href{https://www.cs.cmu.edu/wrc/}{Web Assembly Research Center}}
\bigskip

\noindent {\bf Northeastern University}, \emph{Boston, MA, USA} \vspace{0.01in} \dates{9/2016 -- 7/2018}
\newline\noindent {\bf \em Assistant Clinical Professor}
\newline College of Computer and Information Science
\bigskip

\noindent {\bf Scala Center, EPFL}, \emph{Lausanne, Switzerland} \vspace{0.01in} \dates{10/2015 -- 7/2018}
\newline\noindent {\bf \em Executive Director, Research Scientist}
\newline\noindent Founded a new not-for-profit center dedicated to research,\vspace{-0.03in}
\newline\noindent open source development, and education surrounding the Scala\vspace{-0.03in}
\newline\noindent programming language.
\bigskip

\noindent {\bf Databricks}, \emph{Berkeley, CA, USA} \vspace{0.01in} \dates{8/2014 -- 11/2014}
\newline\noindent {\bf \em Research Intern}
\newline\noindent Supervisor: Matei Zaharia
\newline\noindent Integrated Scala Pickling, our framework for fast, boilerplate-free, extensible\vspace{-0.03in}
\newline\noindent serialization focused on distributed programming (OOPSLA'13), into Spark.\vspace{-0.03in}
\newline\noindent Developed generalization of Spark/MapReduce programming model. (JFP'18).

% \bigskip
\pagebreak

%% Teaching Experience
\medskip
\marginhead{Teaching \newline Experience \newline (Classroom)}

\noindent {\bf Co-Instructor},  \dates{Fall 2020, 2022, 2023, 2024}
\newline\noindent 15-440/15-640: Distributed Systems \dates{Carnegie Mellon}
\bigskip

\noindent {\bf Instructor, Designer},  \dates{Fall 2020, Spring 2021}
\newline\noindent 17-400/17-700: Data Science and Machine Learning at Scale \dates{Carnegie Mellon}
\bigskip

\noindent {\bf Co-Instructor},  \dates{Spring 2020}
\newline\noindent 10-405/10-605: Machine Learning with Large Datasets \dates{Carnegie Mellon}
\bigskip


\noindent {\bf Co-Instructor},  \dates{Spring 2019 \& Spring 2020}
\newline\noindent 17-356: Software Engineering for Startups \dates{Carnegie Mellon}
\bigskip

\noindent {\bf Instructor, Designer},  \dates{Spring 2018}
\newline\noindent CS4240: Large-Scale Parallel Data Processing \dates{Northeastern}
%\newline\noindent Northeastern University senior-level undergraduate course on big data \dates{Northeastern}
%\newline\noindent processing, covering Spark, Hadoop, TensorFlow, amongst others.
%\newline (\textasciitilde40 students)
\bigskip

\noindent {\bf Instructor, Designer}, \dates{Fall 2016}
\newline\noindent CS7680: Programming Models for Distributed Computation \dates{Northeastern}
%\newline\noindent Northeastern University PhD-level course on programming models for \dates{Northeastern}
%\newline\noindent distributed systems. (\textasciitilde20 students)
\bigskip

\noindent {\bf Co-Instructor, Co-Designer}, {(\em with Viktor Kun\v cak \& Martin Odersky)} \dates{Spring 2016}
\newline\noindent CS 206: Parallelism \& Concurrency \dates{EPFL}
\bigskip

\noindent {\bf Co-Instructor, Co-Designer}, {(\em with Viktor Kun\v cak \& Martin Odersky)} \dates{Spring 2015}
\newline\noindent CS 212: Reactive Programming \& Parallelism \dates{EPFL}
\bigskip

\noindent {\bf (Lead) Teaching Assistant}, \dates{Fall 2011-2014}
\newline\noindent CS 201: Functional Programming \dates{EPFL}
%\newline\noindent Required EPFL undergraduate course on functional \& logic programming \dates{EPFL}
%\newline\noindent (\textasciitilde160 students)
\bigskip

%\noindent {\bf Instructor, Co-Designer}, {\em Reactive Programming \& Parallelism} \dates{Spring 2015 \& 2016}
%\newline\noindent EPFL Undergraduate course on parallel, distributed, and asynchronous \dates{EPFL}
%\newline\noindent programming (90 -- 150 students)
%\bigskip


\medskip
\marginhead{Teaching \newline Experience \newline (MOOCs)}

\noindent {\bf Instructor, Designer}, {\em Big Data Analysis with Scala and Spark} \dates{2017 --}
\newline\noindent Popular Coursera MOOC on big data analysis using Spark. \dates{Coursera}
\smallskip
\begin{easylist}[itemize]
& Designed lectures and produced lecture videos. Designed exercises
\newline and developed cloud-hosted automated graders.

& Between March-November 2017, over 120,000 registered learners.
\end{easylist}

%\begin{easylist}[itemize]
%& General introduction to distributed systems for big data
%\newline up through shuffling and optimizations like partitioning,
%\newline as well as the basics of data analytics.
%
%& Between March-November 2017, over 120,000 registered learners.
%\end{easylist}
\bigskip


%\noindent {\bf Instructor, Co-Designer}, {\em Parallel Programming \& Data Analysis} \dates{2015}
%\newline\noindent Upcoming Coursera MOOC on parallel, distributed, and asynchronous
%\newline\noindent programming.
%\bigskip

\noindent {\bf Lead}, {\em Scala Specialization (mini-degree)} \dates{2015 --}
\newline\noindent Responsible for EPFL's offering of a Scala {\em mini-degree} on Coursera. \dates{Coursera}
\smallskip
\begin{easylist}[itemize]
& Assembled offering of 4 Scala MOOCs, topped off with a capstone
\newline project. Taught and produced 1 course in the specialization and
\newline managed the development of the remaining 3 courses and the project.
\end{easylist}
\bigskip


\noindent {\bf Lead}, {\em Functional Programming Principles in Scala} \dates{2012 -- 2014}
\newline\noindent Popular Coursera MOOC on functional programming in Scala. \dates{Coursera}
\smallskip
\begin{easylist}[itemize]
& Lead teaching staff member, organized a team of graduate
\newline students, managed content production, designed course exercises
\newline with cloud-hosted grading, production of lecture videos, etc.

& >400,000 learners across iterations \& largest completion
\newline rate for a course its size (\textasciitilde19\%)
\end{easylist}
\bigskip


%\noindent {\bf Instructor}, {\em Scala as a Research Tool} \dates{2013}
%\newline\noindent ECOOP Tutorial
% \bigskip


% \noindent {\bf Teaching Assistant}, {\em Programming Principles} \dates{2011, 2014}
% \newline\noindent Required EPFL Undergraduate course on functional and logic programming
% \newline\noindent (\textasciitilde160 students)

\bigskip
\marginhead{Book}

\noindent{\bf Distributed Programming}\dates{MIT Press TBD}
\newline\noindent Heather Miller, Nat Dempkowski, James Larisch,
\newline\noindent Christopher Meiklejohn, and Philipp Haller
\smallskip
\newline\noindent A textbook about the building blocks we use to build distributed systems. These range from the small, RPC, futures, actors, to the large; systems built up of these components like MapReduce and Spark. We explore issues and concerns central to distributed systems like consistency, availability, and fault tolerance, from the lens of the programming models and frameworks that the programmer uses to build these systems.
\newline\noindent\href{https://github.com/heathermiller/dist-prog-book}{\em Source (draft)}
\bigskip


\bigskip
%% Publications: Blogs
\marginhead{Publications: \newline Recent Popular \newline Media}
% \medskip

\noindent\href{https://bair.berkeley.edu/blog/2024/02/18/compound-ai-systems/}{\bf The Shift from Models to Compound AI Systems}\dates{Berkeley AI Blog}
\newline\noindent Matei Zaharia, Omar Khattab, Lingjiao Chen, Jared Quincy Davis,\vspace{-0.03in}\dates{(Feb 2024)}  \newline\noindent Heather Miller, Chris Potts, James Zou, Michael Carbin,\vspace{-0.03in}
\newline\noindent Jonathan Frankle, Naveen Rao, Ali Ghodsi
\newline\noindent\emph{Berkeley Artificial Intelligence Research (BAIR) Blog, February 18, 2024}
\bigskip

\noindent\href{https://www.twosigma.com/articles/a-guide-to-large-language-model-abstractions/}{\bf A Guide to Large Language Model Abstractions}\dates{Two Sigma Blog}
\newline\noindent Peter Yong Zhong, Haoze He, Omar Khattab, Christopher Potts,\vspace{-0.03in}\dates{(Jan 2024)}
\newline\noindent Matei Zaharia, Heather Miller
\newline\noindent\emph{Two Sigma Insights, corporate blog, January 16, 2024}
\bigskip



\bigskip
%% Publications: Journals
\marginhead{Publications: \newline Journals}
% \medskip

%% Use revnumerate environment if numbered publications are needed.
%% (Include it above in the preamble).
%% \renewcommand{\labelenumi}{\textsc{a}\theenumi.}
%% \begin{revnumerate}

% \noindent\href{http://infoscience.epfl.ch/record/191239}{\bf Function-Passing Style: Typed, Distributed}\dates{}\vspace{-0.03in}
% \newline\noindent\href{http://infoscience.epfl.ch/record/191239}{\bf Functional Programming}
\noindent{\bf A Reduction Semantics for Direct-Style Asynchronous Observables}\dates{JLAMP 2019}
\newline\noindent Philipp Haller, Heather Miller
\newline\noindent\emph{Journal of Logical and Algebraic Methods in Programming, Volume 105, p75-111.}
\bigskip

\noindent{\bf A Programming Model and Foundation for Lineage-Based Distributed}\dates{JFP 2018}
\newline\noindent{\bf Computation}
\newline\noindent Heather Miller, Philipp Haller, Normen M\"{u}ller
\newline\noindent\emph{Journal of Functional Programming, Volume 28, e7.}
\newline\noindent\emph{Special Issue: Programming Languages for Big Data}
\bigskip



\bigskip
%% Publications: Conferences
\marginhead{Publications: \newline Conferences}

% ICLR
% ASPLOS
% ICSE
% PaPoC 23

\noindent\href{https://openreview.net/forum?id=sY5N0zY5Od}{\bf DSPy: Compiling Declarative Language Model Calls}\dates{ICLR 2024 \textbf{spotlight}}\vspace{-0.03in}
\newline\noindent\href{https://openreview.net/forum?id=sY5N0zY5Od}{\bf into State-of-the-Art Pipelines}
\newline\noindent Omar Khattab, Arnav Singhvi, Paridhi Maheshwari, Zhiyuan Zhang,\vspace{-0.03in} 
\newline\noindent Keshav Santhanam, Sri Vardhamanan A, Saiful Haq, Ashutosh Sharma,\vspace{-0.03in} 
\newline\noindent Thomas T. Joshi, Hanna Moazam, Heather Miller, Matei Zaharia, Christopher Potts
\newline\noindent\emph{International Conference on Learning Representations}
\bigskip

\noindent\href{https://dl.acm.org/doi/10.1145/3620666.3651338}{\bf Flexible Non-intrusive Dynamic Instrumentation for WebAssembly}\dates{ASPLOS 2024}
\newline\noindent Ben L. Titzer, Elizabeth Gilbert, Bradley Wei Jie Teo, Yash Anand,\vspace{-0.03in}
\newline\noindent Kazuyuki Takayama, Heather Miller
\newline\noindent\emph{ACM International Conference on Architectural Support for}\vspace{-0.03in} 
\newline\noindent\emph{Programming Languages and Operating Systems}
\bigskip

\noindent\href{https://dl.acm.org/doi/10.1145/3639478.3640021}{\bf Can My Microservice Tolerate an Unreliable Database?}\dates{ICSE 2024 Demo}\vspace{-0.03in}
\newline\noindent\href{https://dl.acm.org/doi/10.1145/3639478.3640021}{\bf Resilience Testing with Fault Injection and Visualization}
\newline\noindent Michael Assad, Christopher Meiklejohn, Heather Miller, Stephan Krusche
\newline\noindent\emph{IEEE/ACM 46th International Conference on Software Engineering}
\bigskip

\noindent\href{https://dl.acm.org/doi/10.1145/3542929.3563466}{\bf Method overloading the circuit}\dates{SoCC 2022}
\newline\noindent Christopher Meiklejohn, Lydia Stark, Cesare Celozzi, Matt Ranney, Heather Miller
\newline\noindent\emph{ACM Symposium on Cloud Computing}
\bigskip

\noindent\href{https://dl.acm.org/doi/10.1145/3472883.3487005}{\bf Service-Level Fault Injection Testing}\dates{SoCC 2021}
\newline\noindent Christopher Meiklejohn, Andrea Estrada, Yiwen Song, Rohan Padhye, Matt Ranney, Heather Miller
\newline\noindent\emph{ACM Symposium on Cloud Computing}
\bigskip

\noindent\href{https://arxiv.org/abs/2004.04303}{\bf Composing and Decomposing Op-Based CRDTs with}\dates{ICFP 2020}\vspace{-0.03in}
\newline\noindent\href{https://arxiv.org/abs/2004.04303}{\bf Semidirect Products}
\newline\noindent Matthew Weidner, Christopher Meiklejohn, Heather Miller
\newline\noindent\emph{ACM SIGPLAN International Conference on Functional Programming}
\bigskip

\noindent{\bf Heard it Through the Gitvine: An Empirical Study of Tool}\dates{FSE 2020}\vspace{-0.03in}
\newline\noindent{\bf Diffusion Across the npm Ecosystem}
\newline\noindent Hemank Lamba, Asher Trockman, Daniel Armanios, Christian K\"{a}stner, \vspace{-0.03in}
\newline\noindent Heather Miller, Bogdan Vasilescu
\newline\noindent\emph{ACM Symposium on the Foundations of Software Engineering}
\bigskip

\noindent\href{https://www.usenix.org/system/files/atc19-meiklejohn.pdf}{\bf Partisan: Scaling the Distributed Actor Runtime}\dates{USENIX ATC 2019}
\newline\noindent Christopher Meiklejohn, Heather Miller, Peter Alvaro
\newline\noindent\emph{USENIX Annual Technical Conference}
\bigskip

\noindent\href{https://arxiv.org/pdf/1908.07883}{\bf Scala Implicits are Everywhere: A Large-Scale Study of the Use}\dates{OOPSLA 2019}\vspace{-0.03in}
\newline\noindent\href{https://arxiv.org/pdf/1908.07883}{\bf of Implicits in the Wild}
\newline\noindent Filip K\v{r}ikava, Heather Miller, Jan Vitek
\newline\noindent\emph{ACM SIGPLAN Conference on Object Oriented Programming, Systems,}
\newline\noindent\emph{Languages and Applications}
\bigskip

\noindent\href{https://infoscience.epfl.ch/record/229878}{\bf Simplicitly: Foundations and Applications of Implicit Function Types}\dates{POPL 2018}
\newline\noindent Martin Odersky, Olivier Blanvillain, Fengyun Liu, Aggelos Biboudis
\newline\noindent Heather Miller, Sandro Stucki
\newline\noindent\emph{ACM SIGPLAN Symposium on Principles of Programming Languages}
\bigskip

\noindent\href{https://infoscience.epfl.ch/record/205822}{\bf Function Passing: A Model for Typed, Distributed Functional}\dates{SPLASH 2016}\vspace{-0.03in}
\newline\noindent\href{https://infoscience.epfl.ch/record/205822}{\bf Programming}
\newline\noindent Heather Miller, Philipp Haller, Normen M\"{u}ller, Joceyln Boullier
\newline\noindent\emph{ACM SIGPLAN International Symposium on New Ideas, New Paradigms,}
\newline\noindent\emph{and Reflections on Programming \& Software}
\bigskip

\noindent\href{http://infoscience.epfl.ch/record/191239}{\bf Spores: A Type-Based Foundation for Closures in the Age of}\dates{ECOOP 2014}\vspace{-0.03in}
\newline\noindent\href{http://infoscience.epfl.ch/record/191239}{\bf Concurrency and Distribution}
% \smallskip
\newline\noindent Heather Miller, Philipp Haller, Martin Odersky
\newline\noindent\emph{European Conference on Object Oriented Programming}
\bigskip

\noindent\href{http://infoscience.epfl.ch/record/190022}{\bf Functional Programming For All! Scaling a MOOC for Students}\dates{ICSE 2014}\vspace{-0.03in}
\newline\noindent\href{http://infoscience.epfl.ch/record/190022}{\bf And Professionals Alike}
% \smallskip
\newline\noindent Heather Miller, Philipp Haller, Lukas Rytz, Martin Odersky
\newline\noindent\emph{ACM SIGSOFT International Conference on Software Engineering}
\bigskip

\noindent\href{http://infoscience.epfl.ch/record/188383}{\bf Instant Pickles: Generating Object-Oriented Pickler}\dates{OOPSLA 2013}\vspace{-0.03in}
\newline\noindent\href{http://infoscience.epfl.ch/record/188383}{\bf Combinators for Fast and Extensible Serialization}
% \smallskip
\newline\noindent Heather Miller, Philipp Haller, Eugene Burmako, Martin Odersky
\newline\noindent\emph{ACM SIGPLAN Conference on Object Oriented Programming, Systems,}
\newline\noindent\emph{Languages and Applications}
\bigskip


%% Publications: Workshops
\marginhead{Publications: \newline Workshops}

\noindent\href{https://dl.acm.org/doi/10.1145/3578358.3591323}{\bf For-Each Operations in Collaborative Apps}\dates{PaPoC 2023}
\newline\noindent Matthew Weidner, Ria Pradeep, Benito Geordie, Heather Miller
\newline\noindent\emph{Workshop on Principles and Practice of Consistency for Distributed Data}
\bigskip

\noindent\href{https://kilthub.cmu.edu/articles/conference_contribution/Programmer_Experience_When_Using_CRDTs_to_Build_Collaborative_Webapps_Initial_Insights/22277341/1}{\bf Programmer Experience When Using CRDTs to Build}\dates{PLATEAU 2023}\vspace{-0.03in}
\newline\noindent\href{https://kilthub.cmu.edu/articles/conference_contribution/Programmer_Experience_When_Using_CRDTs_to_Build_Collaborative_Webapps_Initial_Insights/22277341/1}{\bf Collaborative Webapps: Initial Insights}
\newline\noindent Yicheng Zhang, Matthew Weidner, Heather Miller
\newline\noindent\emph{Workshop on the Intersection of Human Computer Interaction and}\vspace{-0.03in}\newline\noindent\emph{Programming Languages}
\bigskip

\noindent\href{https://dl.acm.org/citation.cfm?id=3341131}{\bf Checking-in on Network Functions}\dates{ANRW 2019}
\newline\noindent Zeeshan Lakhani, Heather Miller
\newline\noindent\emph{ACM/IRTF Applied Networking Research Workshop}
\bigskip

\noindent\href{https://www.usenix.org/system/files/conference/hotedge18/hotedge18-papers-meiklejohn.pdf}{\bf Towards a Solution to the Red Wedding Problem}\dates{USENIX HotEdge 2018}
\newline\noindent Christopher Meiklejohn, Heather Miller, Zeeshan Lakhani
\newline\noindent\emph{USENIX Workshop on Hot Topics in Edge Computing}
\bigskip

\noindent\href{https://infoscience.epfl.ch/record/205039}{\bf Distributed Programming via Safe Closure Passing}\dates{PLACES 2015}
\newline\noindent Philipp Haller, Heather Miller
\newline\noindent\emph{Programming Language Approaches to Communication}
\newline\noindent\emph{and Concurrency Centric Systems}
\bigskip

\noindent\href{http://infoscience.epfl.ch/record/188383}{\bf RAY: Integrating Rx and Async for Direct-Style Reactive Streams}\dates{REM 2013}
% \smallskip
\newline\noindent Philipp Haller, Heather Miller
\newline\noindent\emph{ACM SPLASH Workshop on Reactivity, Events and Modularity}
\bigskip

\noindent\href{http://infoscience.epfl.ch/record/180265}{\bf FlowPools: A Lock-Free Deterministic Concurrent}\dates{LCPC 2012}\vspace{-0.03in}
\newline\noindent\href{http://infoscience.epfl.ch/record/180265}{\bf Dataflow Abstraction}
% \smallskip
\newline\noindent Aleksandar Prokopec, Heather Miller, Tobias Schlatter,
\newline\noindent Philipp Haller, Martin Odersky
\newline\noindent\emph{International Workshop on Languages and Compilers for Parallel Computing}
\vspace{0.03in}
\newline\noindent {\small Invited to Revised Selected Papers on the 25th International Workshop on}
\vspace{-0.03in}
\newline\noindent {\small Languages and Compilers for Parallel Computing, Lecture Notes in Computer}
\vspace{-0.03in}
\newline\noindent {\small Science, Vol. 7760, 2013}
\bigskip

\noindent\href{http://infoscience.epfl.ch/record/170032}{\bf Tools and Frameworks for Big Learning in Scala: Leveraging the}\dates{BigLearn 2011}\vspace{-0.03in}
\newline\noindent\href{http://infoscience.epfl.ch/record/170032}{\bf Language for High Productivity and Performance}
% \smallskip
\newline\noindent Heather Miller, Philipp Haller, Martin Odersky
\newline\noindent\emph{NIPS Workshop on Parallel and Large-Scale Machine Learning}
\bigskip

\noindent\href{http://infoscience.epfl.ch/record/165111}{\bf Parallelizing Machine Learning -- Functionally: A Framework}\dates{Scala 2011}\vspace{-0.03in}
\newline\noindent\href{http://infoscience.epfl.ch/record/165111}{\bf and Abstractions for Parallel Graph Processing}
% \smallskip
\newline\noindent Philipp Haller, Heather Miller
\newline\noindent\emph{Scala Workshop}
\bigskip

% %% In Progress
% \marginhead{{\vskip 0.4em}Submitted/In Preparation}
% \medskip

% %% Use revnumerate environment if numbered publications are needed.
% %% (Include it above in the preamble).
% %% \renewcommand{\labelenumi}{\textsc{a}\theenumi.}
% %% \begin{revnumerate}


% \noindent{\bf Monotonicity Types}\dates{}
% \newline\noindent Kevin Clancy, Heather Miller, Christopher Meiklejohn
% \medskip

% \noindent{\bf The Essence of Coordination-Free Distributed Computation}\dates{}
% \newline\noindent Christopher Meiklejohn, Kevin Clancy, Heather Miller
% \medskip

%% Selected Tech Reports
%\bigskip
\marginhead{{\vskip 0.3em}Selected \newline Tech Reports}
 \medskip

%% Use revnumerate environment if numbered publications are needed.
%% (Include it above in the preamble).
%% \renewcommand{\labelenumi}{\textsc{a}\theenumi.}
%% \begin{revnumerate}

\noindent\href{https://infoscience.epfl.ch/record/221395}{\bf The Function Passing Model: Types, Proofs, and Semantics}\dates{May 2016}
\newline\noindent Philipp Haller, Normen M\"{u}ller, Heather Miller
\medskip

\noindent{\bf Specialising Parsers for Queries}\dates{April 2016}
\newline\noindent Manohar Jonnalagedda, Jorge Vicente Cantero, Heather Miller, Martin Odersky
\medskip

\noindent\href{https://infoscience.epfl.ch/record/197948}{\bf Improving Human-Compiler Interaction Through Customizable} \dates{December 2014}\vspace{-0.03in}
\newline\noindent\href{https://infoscience.epfl.ch/record/197948}{\bf Type Feedback}\dates{}
\newline\noindent Hubert Plociniczak, Heather Miller, Martin Odersky
\medskip

\noindent\href{https://infoscience.epfl.ch/record/199389}{\bf Self-Assembly: Lightweight Language Extension and Datatype}\dates{August 2014}\vspace{-0.03in}
\newline\noindent\href{https://infoscience.epfl.ch/record/199389}{\bf Generic Programming, All-in-One!}\dates{}
\newline\noindent Heather Miller, Philipp Haller, Bruno C. d. S. Oliveira
\medskip

\noindent\href{http://infoscience.epfl.ch/record/191240}{\bf Spores, Formally}\dates{December 2013}
% \smallskip
\newline\noindent Heather Miller, Philipp Haller
% \bigskip
\medskip

\noindent\href{http://infoscience.epfl.ch/record/181098}{\bf  FlowPools: A Lock-Free Deterministic Concurrent Dataflow}\dates{June 2012}\vspace{-0.03in}
\newline\noindent\href{http://infoscience.epfl.ch/record/181098}{\bf  Abstraction -- Proofs}\dates{}
% \smallskip
\newline\noindent Aleksandar Prokopec, Heather Miller, Philipp Haller


%% External Service
\bigskip
\marginhead{External \newline Service}

% \noindent {\bf Committees:}

\noindent {\bf General Chair and/or Program Chair:}
%\smallskip
\newline\noindent {\em \href{https://sites.google.com/view/compound-ai-systems-workshop/home}{Compound AI Systems Workshop}} {(\textbf{Compound AI Systems})} \dates{2024}
\newline\noindent {\em ICSE Software Engineering in Practice} {(\textbf{ICSE SEIP})} \dates{2022}
\newline\noindent {\em Curry On} {(\textbf{Curry On})} \dates{2015, 2016, 2017, 2018, 2019}
\newline\noindent {\em Workshop on Principles and Practice of Consistency for Distributed Data} {(\textbf{PaPoC})} \dates{2019}
\newline\noindent {\em Trends in Functional Programming in Education} {(\textbf{TFPIE})} \dates{2018}
\newline\noindent {\em Scala Symposium} {(\textbf{Scala})}  \dates{2013, 2014, 2017}
\newline\noindent {\em Programming Models \& Languages for Distributed Computation} {(\textbf{PMLDC})}  \dates{2016, 2017}
\bigskip

\noindent {\bf Organizing Committee Member:}
%\smallskip
\newline {\em Object-Oriented Programming, Systems, Languages \& Applications} {(\textbf{OOPSLA})} \dates{2018}
\newline {\em European Conference on Object-Oriented Programming} {(\textbf{ECOOP})} \dates{2015 -- 2019}
\bigskip

\noindent {\bf Program Committee Member:}
%\smallskip
\newline\noindent {\em International Conference on Software Engineering} {(\textbf{ICSE})} \dates{2021}
\newline\noindent {\em USENIX Workshop on Hot Topics in Cloud Computing} {(\textbf{USENIX HotCloud})} \dates{2020}
\newline\noindent {\em USENIX Workshop on Hot Topics in Edge Computing} {(\textbf{USENIX HotEdge})} \dates{2020}
\newline\noindent {\em Workshop on Principles and Practice of Consistency for Distributed Data} {(\textbf{PaPoC})} \dates{2020}
\newline\noindent {\em Object-Oriented Programming, Systems, Languages \& Applications} {(\textbf{OOPSLA})} \dates{2019}
\newline\noindent {\em European Conference on Object-Oriented Programming} {(\textbf{ECOOP})} \dates{2019}
\newline\noindent {\em Symposium on Principles of Programming Languages} {(\textbf{POPL})} \dates{2019}
\newline\noindent {\em International Conference on Functional Programming} {(\textbf{ICFP})} \dates{2018}
\newline\noindent {\em Off the Beaten Track} {(\textbf{OBT})} \dates{2018}
\newline\noindent {\em Object-Oriented Programming, Systems, Languages \& Applications} {(\textbf{OOPSLA})} \dates{2017}
\newline\noindent {\em Scala Symposium} {(\textbf{Scala})} \dates{2016}
\newline\noindent {\em Symposium on Trends in Functional Programming} {(\textbf{TFP})} \dates{2016}
\newline\noindent {\em Software Language Engineering} {(\textbf{SLE})} \dates{2016}
\newline\noindent {\em Symposium on Applied Computing} {(\textbf{SAC})} \dates{2016}
\newline\noindent {\em Programming Language Evolution} {(\textbf{PLE})} \dates{2015}
\newline\noindent {\em Domain-Specific Language Design and Implementation} {(\textbf{DSLDI})} \dates{2015}
\medskip

\noindent {\bf External Review Committee Member:}
\newline\noindent {PLDI 2020}, {PLDI 2018}, {ECOOP 2016}, {ECOOP 2013}, {Scala 2013}
\medskip
\newline\noindent {\bf Artifact Evaluation Committee:} {POPL 2015}
\medskip

%% Diversity and Outreach
\bigskip
\marginhead{Diversity \&\newline Outreach}

%\noindent {\bf Girls Code It: Intensive Pre-College Computer Science Program} \dates{Summer 2018}
%\newline\noindent Conceived of and am organizing large pre-college program aimed at \dates{Northeastern}
%\newline\noindent preparing high school-aged girls for a career in Computer Science.
%\smallskip
%\newline\noindent 6 week-long residential program for 100 students which awards
%\newline\noindent college credit and puts alumni of the program on an accelerated
%\newline\noindent CS track upon matriculating at Northeastern University.
%\bigskip

\noindent {\bf Confluence Talks Co-Creator/Organizer}
\newline\noindent Co-created a new talk series at CMU intent on building a bridge between
\newline\noindent Pittsburgh's local tech scene and industry-relevant research at CMU.

\medskip

\noindent {\bf ScalaBridge Organizer}
\newline\noindent Organizer of free full-day workshops on the weekends aimed at teaching women
\newline\noindent and underrepresented minorities in computing how to think computationally and
\newline\noindent how to program in Scala.
\smallskip
\newline\noindent {\em ScalaBridge Chapters: Basel (CH), Z\"{u}rich (CH), Copenhagen (DK), Boston (US).}
\bigskip

%% Open Source
\medskip
\marginhead{{\vskip 0.1em}Open Source}

\vspace{0.01in}
\noindent {\bf Scala Programming Language}, {\em member of the Scala team} \dates{2011 --}

\vspace{0.05in}
\begin{easylist}[itemize]
& \href{http://docs.scala-lang.org/sips/pending/spores.html}{{\bf Scala Spores} (Scala Improvement Proposal SIP-21)}, {\bf \em project lead}
\newline novel type-based abstraction for using closures safely
\newline in concurrent and distributed environments

& \href{http://lampwww.epfl.ch/~hmiller/pickling/}{{\bf Scala Pickling}}, {\bf \em project lead}
\newline novel framework for fast, boilerplate-free, extensible serialization.
\newline Adopted by sbt, the most widely-used build tool for Scala. Popular
\newline open-source project on GitHub with >820 stars \& dozens of contributors

& \href{http://docs.scala-lang.org/sips/completed/futures-promises.html}{{\bf Scala Futures \& Promises} (Scala Improvement Proposal SIP-14)}, {\bf \em team member}
\newline unified non-blocking concurrency substrate for
\newline Scala, Akka, Play, and others

& \href{http://docs.scala-lang.org/}{{\bf Scala Documentation}}, {\bf \em creator, writer, lead maintainer}
\newline a central website for community-driven documentation for
\newline the Scala programming language and core libraries

& \href{https://wiki.scala-lang.org/display/SW/Scaladoc}{{\bf Scaladoc}}, {\bf \em co-maintainer}
\newline documentation tool for Scala's official API documentation

\end{easylist}

\bigskip

%% Honors
\medskip
\marginhead{Honors}

\noindent \href{https://sites.google.com/aito.org/home/aito-dahl-nygaard/2023-winners}{Dahl-Nygaard Junior Prize} \dates{2023}
\newline\noindent ACM SIGPLAN Programming Languages Software Award (for Scala) \dates{2019}
\newline\noindent US National Science Foundation Graduate Research Fellowship \dates{2011 -- 2014}
% \newline\noindent PLMW Travel Grant \dates{2014-2015}
% \newline\noindent ICSE Travel Grant \dates{2014}
\newline\noindent EPFL Outstanding Teaching Award \dates{2012}
\newline\noindent EPFL Computer Science Fellowship \dates{2009 -- 2010}
\newline\noindent Most Outstanding Audio Engineering Student, University of Miami \dates{2009}
\newline\noindent Most Outstanding Eta Kappa Nu Student, University of Miami \dates{2009}
\newline\noindent Information Technology Scholarship, University of Miami \dates{2006 -- 2009}
\newline\noindent John Farina Family Scholarship, University of Miami \dates{2006 -- 2009}
\newline\noindent Eta Kappa Nu \dates{2008}
\newline\noindent Tau Beta Pi \dates{2008}
\newline\noindent SMART US Department of Defense Scholarship Alternate \dates{2007}
\newline\noindent Cooper Union Full Tuition Scholarship \dates{2004 -- 2006}

\bigskip

%\pagebreak

%% Talks
\medskip
\marginhead{Selected Talks}

\noindent{\bf Open Source Numbers Everybody Should Know}\dates{Open Source Summit North America}\vspace{-0.03in}
\linebreak\noindent Austin TX, USA (held virtually). June 29, 2020 \dates{{\bf \em (keynote)}}
\bigskip


\noindent{\bf Open Source Numbers Everybody Should Know}\dates{BOBKonf 2020}\vspace{-0.03in}
\linebreak\noindent Berlin, Germany. February 28, 2020 \dates{{\bf \em (keynote)}}
\bigskip

\vspace{-0.02in}
\noindent{\bf The Times They Are a-Changin’: A Data-Driven Portrait of} \dates{Scale By the Bay 2019}\vspace{-0.03in} 
\newline\noindent{\bf New Trends in How We Build Software, Open Source,}\dates{{\bf \em (keynote)}}\vspace{-0.03in}
\newline\noindent {\bf \& What Even is Entry-Level Now}\dates{} \vspace{-0.03in}
\linebreak\noindent Oakland, CA, USA. November 14, 2019
\bigskip

\noindent{\bf Scala Implicits are Everywhere: A Large-Scale Study of the Use}\dates{OOPSLA 2019}\vspace{-0.03in}
\linebreak\noindent Athens, Greece. October 24, 2019 \dates{}
\bigskip

\noindent{\bf We're Building On Hollowed Foundations: Worrying Trends in} \dates{Programming 2019}\vspace{-0.03in}
\newline\noindent {\bf Open Source and What We Can Actually Do About It}\dates{{\bf \em (keynote)}}
\linebreak\noindent Genoa, Italy. April 4, 2019
\bigskip

\noindent{\bf Towards Language Support for Distributed Systems} \dates{Code Mesh 2018}\vspace{-0.03in}
\linebreak\noindent London, UK. November 9, 2018 \dates{{\bf \em (invited)}}
\bigskip

\noindent{\bf What Happened to Distributed Programming Languages?} \dates{SPLASH-I 2018}\vspace{-0.03in}
\newline\noindent {Boston, MA, USA. November 6, 2018} \dates{{\bf \em (invited)}}
\bigskip

\noindent{\bf Towards Language Support for Distributed Systems} \dates{Strange Loop 2018}\vspace{-0.03in}
\linebreak\noindent St. Louis, MO, USA. September 27, 2018
\bigskip

\noindent{\bf I'm a Young Assistant Professor: AMA. + Heather's Unsolicited} \dates{PLMW 2018}\vspace{-0.03in}
\newline\noindent {\bf  Advice About Grad School}\dates{{\bf \em (invited)}}
\linebreak\noindent St. Louis, MO, USA. September 23, 2018
\bigskip

\noindent{\bf We're Building On Hollowed Foundations: Worrying Trends in} \dates{Lambda Days 2018}\vspace{-0.03in}
\newline\noindent {\bf Open Source and What You Can Actually Do About It}\dates{{\bf \em (keynote)}}
\linebreak\noindent Krakow, Poland. February 22, 2018
\bigskip

\noindent{\bf The Dramatic Consequences of the Open Source Revolution:}\dates{Devoxx 2017}\vspace{-0.03in}
\newline\noindent {\bf Unrecognized Challenges \& Some Modest Attempts at}\dates{{\bf \em (invited)}}\vspace{-0.03in}
\newline\noindent {\bf Solutions in Scala}\dates{}
\linebreak\noindent Paris, France. April 7, 2017
\bigskip

\noindent{\bf The Dramatic Consequences of the Open Source Revolution}\dates{Scala Exchange 2016}\vspace{-0.03in}
\newline\noindent {\bf \& How the Scala Center Hopes to Help}\dates{{\bf \em (keynote)}}
\linebreak\noindent London, UK. December 9, 2016
\bigskip


\noindent{\bf Function Passing: A Model for Typed, Distributed Functional}\dates{SPLASH 2016}\vspace{-0.03in}
\newline\noindent {\bf Programming}\dates{}
\linebreak\noindent Amsterdam, The Netherlands. November 2, 2016
\bigskip

\noindent{\bf Introducing the Scala Center}\dates{Scala Days 2016}
\linebreak\noindent New York, NY, US. May 10, 2016 \& Berlin, Germany. June 16, 2016 \dates{{\bf \em (keynote)}}
\newline\noindent {\em (total \textasciitilde1700 attendees)}
\bigskip


\noindent\href{https://speakerdeck.com/heathermiller/function-passing-style-typed-distributed-functional-programming}{\bf Function Passing Style: Typed, Distributed} \dates{Strange Loop 2014}\vspace{-0.03in}
\linebreak\noindent\href{https://speakerdeck.com/heathermiller/function-passing-style-typed-distributed-functional-programming}{\bf Functional Programming}\dates{}
\linebreak\noindent St. Louis, MO, USA. September 19, 2014
\bigskip


\noindent\href{https://speakerdeck.com/heathermiller/spores-a-type-based-foundation-for-closures-in-the-age-of-concurrency-and-distribution}{\bf Spores: A Type-Based Foundation for Closures in the Age of} \dates{ECOOP 2014}\vspace{-0.03in}
\linebreak\noindent\href{https://speakerdeck.com/heathermiller/spores-a-type-based-foundation-for-closures-in-the-age-of-concurrency-and-distribution}{\bf Concurrency and Distribution}\dates{}
\linebreak\noindent Uppsala, Sweden. August 1, 2014
\bigskip

\noindent{\bf Functional Programming For All! Scaling a MOOC for} \dates{ICSE 2014}\vspace{-0.03in}
\linebreak\noindent{\bf Students and Professionals Alike}\dates{}
\linebreak\noindent Hyderabad, India. June 4, 2014
\bigskip

\noindent{\bf Academese to English: Scala's Type System, Dependent Types} \dates{NEScala 2014}\vspace{-0.03in}
\linebreak\noindent{\bf and What It Means To You}\dates{}
\linebreak\noindent New York, NY, USA. March 1, 2014
\bigskip

\noindent\href{https://speakerdeck.com/heathermiller/instant-pickles-generating-object-oriented-pickler-combinators-for-fast-and-extensible-serialization}{\bf Instant Pickles: Generating Object-Oriented Pickler} \dates{OOPSLA 2013}\vspace{-0.03in}
\linebreak\noindent\href{https://speakerdeck.com/heathermiller/instant-pickles-generating-object-oriented-pickler-combinators-for-fast-and-extensible-serialization}{\bf Combinators for Fast and Extensible Serialization}\dates{}
\linebreak\noindent Indianapolis, IN, USA. October 30, 2013
\bigskip

\noindent\href{http://heather.miller.am/files/IU-PL-Abstractions-for-Dist-Programming.pdf}{\bf PL Abstractions for Distributed Programming:} \dates{Indiana University {\bf \em (invited)}}\vspace{-0.03in}
\linebreak\noindent\href{http://heather.miller.am/files/IU-PL-Abstractions-for-Dist-Programming.pdf}{\bf Pickle Your Spores!}\dates{}
\linebreak\noindent Bloomington, IN, USA. October 25, 2013
\bigskip

\noindent\href{https://speakerdeck.com/heathermiller/spores-distributable-functions-in-scala}{\bf Spores: Distributable Functions in Scala} \dates{Strange Loop 2013}
\linebreak\noindent St. Louis, MO, USA. September 19, 2013
\bigskip

\noindent\href{http://heather.miller.am/files/LaME2013-Dataflow.pdf}{\bf Open Issues in Dataflow Programming} \dates{LaME 2013 {\bf \em (invited)}}
\linebreak\noindent Montpellier, France. July 1, 2013
\bigskip

\noindent{\bf Scala as a Research Tool} \dates{ECOOP 2013 Tutorial}
\linebreak\noindent Montpellier, France. July 1, 2013
\bigskip

\noindent\href{https://speakerdeck.com/heathermiller/on-pickles-and-spores-improving-support-for-distributed-programming-in-scala}{\bf On Pickles \& Spores: Improving Scala's Support} \dates{ScalaDays 2013}\vspace{-0.03in}
\linebreak\noindent\href{https://speakerdeck.com/heathermiller/on-pickles-and-spores-improving-support-for-distributed-programming-in-scala}{\bf for Distributed Programming}\dates{}
\linebreak\noindent New York, NY, USA. June 12, 2013
\bigskip

\noindent\href{http://lampwww.epfl.ch/~hmiller/files/Futures-Try-PhillyETE.pdf}{\bf Futures \& Promises in Scala 2.10} \dates{PhillyETE 2013 {\bf \em (invited)}}
\linebreak\noindent Philadelphia, PA, USA. April 2, 2013
\bigskip

%\bigskip
\noindent {\em I am also a frequent speaker in industry, at industrial conferences, developer ``meet-ups'', and everything in between. Some such events include:}
\medskip
\newline\noindent
{\bf Scala Italy} (9/2018, Florence, Italy),
{\bf LxScala} (6/2018, Lisbon, Portugal),
{\bf Open Source Summit} (12/2017, Paris, France),
{\bf Scala World} (9/2017, Lake District, UK),
{\bf \href{https://youtu.be/17yy5BwIiTw}{LxScala}} (5/2017, Lisbon, Portugal),
{\bf Lambda Days} (2/2017, Krakow, Poland),
{\bf\href{https://www.youtube.com/watch?v=67UNErFdr64}{PhillyETE}} (4/2016, Philadelphia, USA),
{\bf Code Mesh} (11/2015, London, UK),
{\bf Scalar} (4/2015, Warsaw, Poland),
{\bf\href{http://fby.by/}{f(by)}} (11/2014, Minsk, Belarus),
{\bf\href{https://www.youtube.com/watch?v=4obTnLVXQWY}{SF Scala}} (11/2014, SF, USA),
{\bf\href{http://www.scalapeno.org.il/#!heather-miller/cj0q}{Scalape\~{n}o}} (9/2014, Tel Aviv, Israel),
{\bf \href{https://www.eventbrite.com/e/soundcloud-techtalks-unconventional-thinking-in-design-and-programming-tickets-12166429117}{SoundCloud TechTalks}} (7/2014, Berlin, Germany),
{\bf Scala Days} (6/2014, Berlin, Germany),
{\bf\href{http://www.nescala.org/2014}{NEScala}} (3/2014, NYC, USA), amongst others.

\bigskip


% %% Selected Broader Service
% \medskip
% \marginhead{Selected \newline Broader \newline Service}

% \noindent \href{http://ic.epfl.ch/conseil-de-faculte}{\bf EPFL Computer Science Faculty Council}, {\bf \em PhD Student Representative} \dates{2012 --}
% \newline\noindent Members include the dean of the faculty as well as representatives
% \newline\noindent from every branch of the faculty, administrative, PhD, faculty, etc.
% \newline\noindent Quarterly meetings to steer the faculty and introduce new initiatives.
% \bigskip

% \noindent \href{http://ic-gsa.epfl.ch/}{\bf EPFL CS Graduate Student Association}, {\bf \em President} \dates{2009 -- 2011}
% \newline\noindent Volunteer student organization with a mission to foster a sense of
% \newline\noindent community and collaboration between different research groups in
% \newline\noindent the faculty. Initiatives led/introduced:
% \vspace{0.05in}
% \begin{easylist}[itemize]
% & {\bf Research Day}: college-wide showcase of labs' research activities
% & {\bf PhD Student Open House}: main recruiting event for CS doctoral program
% & {\bf Social Events}: aper\'{o}s, ski trips, outings
% \end{easylist}
% \bigskip

% \noindent {\bf EPFL CS Graduate Student Mentor} \dates{2010 -- 2012}
% \newline\noindent One-on-one mentoring of incoming doctoral students, aided students in
% \newline\noindent integrating into EPFL's research environment and Switzerland as a whole.
% \vspace{0.05in}
% \bigskip


%% External Activities
\medskip
\marginhead{External \newline Activities}

% \noindent {\bf \href{https://www.hackerschool.com/}{Hacker School}}, resident\dates{2015}
% \newline
\noindent {\bf \href{http://scalawags.tv/}{Scalawags Monthly Podcast}}, co-host\dates{2014 -- 2016}

\bigskip


%% Students Supervised
\bigskip
\marginhead{Students \newline Supervised}
%\footnotetext[1]{At EPFL, research groups offer substantial projects for B.Sc./M.Sc. students to complete for credit. EPFL PhD students design and supervise these projects, as well as M.Sc. thesis projects.}

\noindent {\bf \href{https://gulang2019.github.io/}{Siyan Chen}} (co-advised with Phil Gibbons and Ben L. Titzer) \dates{2023 --}
\newline\noindent PhD thesis \dates{Carnegie Mellon}
\medskip

\noindent {\bf Peter Yong Zhong}\dates{2023 --}
\newline\noindent PhD thesis \dates{Carnegie Mellon}
\medskip

\noindent {\bf Haoze Hector He}\dates{2023 --}
\newline\noindent PhD thesis \dates{Carnegie Mellon}
\medskip

\noindent {\bf Elizabeth Gilbert} (co-advised with Ben L. Titzer) \dates{2022 --}
\newline\noindent PhD thesis \dates{Carnegie Mellon}
\medskip

\noindent {\bf \href{https://mattweidner.com/}{Matthew Weidner}} \dates{2019 --}
\newline\noindent {\em Increasing the Flexibility of Collaborative Data Structures} \dates{Carnegie Mellon}
\newline\noindent PhD thesis
\medskip

\noindent {\bf \href{https://christophermeiklejohn.com/}{Dr. Christopher Meiklejohn}} \dates{2018 -- 2024} 
\newline\noindent {\em Resilient Microservice Applications, By Design, and without the Chaos}\dates{Carnegie Mellon}
\newline\noindent PhD thesis 
\medskip

\noindent {\bf Joceyln Boullier}, {\em Evaluating the Efficiacy of the Function Passing Model} \dates{2/2016 -- 8/2016}
\newline\noindent M.Sc. thesis \dates{EPFL}
\medskip

\noindent {\bf Jorge Vicente Cantero}, {\em Implementing the Function Passing Model} \dates{2/2016 -- 6/2016}
\newline\noindent B.Sc. thesis \dates{EPFL}
\medskip

%\noindent {\bf Louis Bliss}, {\em Incremental Picklers for Scala Pickling} \dates{9/2013 -- 1/2014}
%\newline\noindent M.Sc. level, co-supervision with Philipp Haller \dates{EPFL}
%\medskip

\noindent {\bf Thadd\'{e}e Yann Tyl}, {\em Learning Scala Style} \dates{2/2013 -- 6/2013}
\newline\noindent M.Sc. thesis \dates{EPFL}
\medskip

%\noindent {\bf Tobias Schlatter}, {\em FlowSeqs: Barrier-Free ParSeqs} \dates{9/2012 -- 1/2013}
%\newline\noindent M.Sc. level, co-supervision w/ Philipp Haller \& Aleksandar Prokopec \dates{EPFL}
%\medskip
%
%\noindent {\bf Tobias Schlatter}, {\em Multi-Lane FlowPools} \dates{2/2012 -- 6/2012}
%\newline\noindent M.Sc. level, co-supervision w/ Philipp Haller \& Aleksandar Prokopec \dates{EPFL}
%\medskip
%
%\noindent {\bf Pierre Grydbeck}, {\em Parallel Machine Learning: An Expectation} \dates{2/2012 -- 6/2012}
%\newline\noindent {\em Maximization Algorithm for Gaussian Mixture Models}
%\newline\noindent M.Sc. level, co-supervision with Philipp Haller \dates{EPFL}
%\medskip
%
%\noindent {\bf Bruno Studer}, {\em Parallel Machine Learning: Collaborative Filtering} \dates{2/2012 -- 6/2012}
%\newline\noindent {\em via Alternating Least Squares}
%\newline\noindent B.Sc. level, co-supervision with Philipp Haller \dates{EPFL}
%\medskip
%
%\noindent {\bf Stanislav Peshterliev}, {\em Parallel Natural Language Processing} \dates{9/2011 -- 1/2012}
%\newline\noindent {\em Algorithms in Scala}
%\newline\noindent M.Sc. level, co-supervision with Philipp Haller \dates{EPFL}
%\medskip
%
%\noindent {\bf Olivier Blanvillain \& Louis Bliss}, {\em Parallelization of a Collaborative} \dates{9/2011 -- 1/2012}
%\newline\noindent {\em Filtering Algorithm with Menthor}
%\newline\noindent B.Sc. level, co-supervision with Philipp Haller \dates{EPFL}
%\medskip
%
%\noindent {\bf Florian Gysin}, {\em Improving Parallel Graph Processing Through} \dates{9/2011 -- 1/2012}
%\newline\noindent {\em the Introduction of Parallel Collections}
%\newline\noindent M.Sc. level, co-supervision with Philipp Haller \dates{EPFL}
%\medskip
%
%\noindent {\bf Georges Discry}, {\em Extending the Menthor Framework for Parallel} \dates{2/2011 -- 6/2011}
%\newline\noindent {\em Graph Processing to Distributed Computing}
%\newline\noindent M.Sc. level, co-supervision with Philipp Haller \dates{EPFL}
%\medskip

%% References

\bigskip

\marginhead{References}


\noindent\begin{tabular}{lr}

\begin{minipage}[t]{2.5in}
\noindent {\bf Martin Odersky}, Professor
%\newline\noindent {\em Faculty of Computer, Communication, and Information Science}
\newline\noindent {\em \'{E}cole Polytechnique F\'{e}d\'{e}rale de Lausanne}
\newline\noindent \Telefon~+41 21 693 68 63
\newline\noindent \Letter~\href{mailto:martin.odersky@epfl.ch}{martin.odersky@epfl.ch}
\medskip
\end{minipage}
&
\begin{minipage}[t]{2.5in}
\noindent {\bf Matthias Felleisen}, Trustee Professor
%\newline\noindent {\em College of Communication and Information Science}
\newline\noindent {\em Northeastern University}
\newline\noindent \Telefon~+1-617-373-2085
\newline\noindent \Letter~\href{mailto:matthias@ccs.neu.edu}{matthias@ccs.neu.edu}
\medskip
\end{minipage}
\\
\\
\begin{minipage}[t]{2.5in}
\noindent {\bf Matei Zaharia}, Associate Professor
%\newline\noindent {\em Department of Computer Science}
\newline\noindent {\em UC Berkeley}
\newline\noindent \Telefon~+1-510-610-0001
\newline\noindent \Letter~\href{mailto:matei@berkeley.edu}{matei@berkeley.edu}
\medskip
\end{minipage}
&
\begin{minipage}[t]{2.5in}
\noindent {\bf Philipp Haller}, Associate Professor
%\newline\noindent {\em School of Computer Science and Communication}
\newline\noindent {\em KTH Royal Institute of Technology}
\newline\noindent \Telefon~+46 70 738 28 43
\newline\noindent \Letter~\href{mailto:phaller@kth.se}{phaller@kth.se}
\end{minipage}

\end{tabular}





\end{document}
